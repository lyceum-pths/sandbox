\section{Изолированный проводник. Поле внутри и вне его}
	
	Проводник\index{Проводник} -- вещество, в~котором есть достаточно свободных зарядов. \par
	Возьмем изолированный проводник\index{Проводник!изолированный} (в~том смысле, что электроны\index{Электрон} его не~покидают) и~поместим его во~внешнее поле\index{Поле!внешнее}. Поскольку заряды свободны\index{Заряд!свободный}, то~они начнут перемещаться по~проводнику. Проводник в~результате поляризуется\index{Поляризация} и~сам создает поле\index{Поле}. Поток зарядов в~нем прекратится, когда внутреннее поле\index{Поле!внутреннее} скомпенсирует внешнее. Таким образом, проводником будем называть вещество, в~котором всегда достаточно заряда, чтобы скомпенсировать внутри себя любое внешнее поле:
		$$\vec{E}_{\text{внутр}}=\vec{E}_{\text{внешн}}+\vec{E}_{\text{комп}}=0, \quad \vec{F}=q\vec{E}=0.$$
	По крайней мере один электрон\index{Электрон} оторвется от каждого атома. Между тем, телефон А.~М. несет заряд в 100000 кулон. \par
	В проводнике происходит поляризация, и, спустя малое время релаксации\index{Релаксация}, любое поле внутри проводника оказывается скомпенсированным. Итак, основное свойство проводника --
		\begin{equation}
			\vec{E}_{\text{внутр}}=0.
		\end{equation}
	Следствия:
		\begin{enumerate}
			\item Проводник -- эквипотенциальный\index{Эквипотенциал} объем:
					$$\varphi_1-\varphi_2=\int_1^2 E_l\d{l}=0, \quad \varphi_1=\varphi_2.$$
			\item Поле вне проводника перпендикулярно его поверхности (потому что он -- эквипотенциальный объем).
			\item Заряд\index{Заряд} в веществе проводника находится только на его поверхности, а внутри проводника заряда нет. Действительно, рассмотрим объем внутри проводника и запишем для него первое уравнение Максвелла\index{Уравнение!Максвелла!первое}:
				$$0=\oint E_n\d{S}=\frac{q}{\varepsilon_0},$$
			откуда $q=0$.
			\item Если внутри проводника есть полость, то поле этой полости никак не зависит от внешних зарядов и полей. Если в самой полости нет заряда, то поле в ней равно нулю.
		\end{enumerate}