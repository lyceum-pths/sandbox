\section{Уравнения электростатики}

	Закон Кулона \index{Закон!Кулона} и~принцип суперпозиции \index{Принцип!суперпозиции} позволяет подсчитать поле \index{Поле} и~силу в~любой точке пространства. Альтернативный способ сделать это --- уравнения Максвелла \index{Уравнение!Максвелла}. \par
	Вспомним, что электрическое поле\index{Поле!электрическое} центрально симметрично\index{Поле!центрально симметричное} и~консервативно \index{Поле!консервативное}:
		$$A=\oint \vec{F} \dvec\vec{l}=q\oint E_l\d{l}=0,$$
	тогда запишем равенство, называемое \i{вторым уравнением Максвелла}\index{Уравнение!Максвелла!второе}:
		\begin{equation}
			\oint E_l\d{l} =0.
		\end{equation}
	\i{\b{Циркуляция \index{Циркуляция} электростатического поля равна нулю, что отражает консервативность этого поля}}. Это значит, что \, замкнутых силовых линий нет.\par
	Докажем формулу Гаусса-Остроградского \index{Формула!Гаусса-Остроградского} (первое уравнение Максвелла) \index{Уравнение!Максвелла!первое}.\par
	Из каждой замкнутой поверхности\index{Поверхность!замкнутая} выходит число силовых линий\index{Линия!силовая}, пропорциональное суммарному заряду\index{Заряд}:
		$$N\prop q_{\Sigma}.$$
	Тогда
		$$\Gamma=\frac{dN}{dS}\prop E, \quad E\, dS\prop dN.$$
	Величину 
		$$\dvec{\Phi_E}=\vec{E}\d S$$
	назовем потоком\index{Поток!электрического поля} электрического поля. Для произвольной поверхности поток электрического поля сквозь нее равен
		\begin{equation}
			\Phi_E=\int\limits_S E_n\d{S}.
		\end{equation}
	Таким образом мы~определили понятие, аналогичное интуитивному понятию густоты\index{Густота}. Теперь
		\begin{equation}\label{eq:prop1}
			N\prop\oint E\d{S}\prop q_{\Sigma}.
		\end{equation}
	\i{\b{Поток\index{Поток!электрического поля} электрического поля через замкнутую поверхность\index{Поверхность!замкнутая} пропорционален суммарному заряду внутри поверхности.}} Поток считается положительным, если поле идет наружу.\par
	Запишем~(\ref{eq:prop1}) с коэффициентом пропорциональности:
		\begin{equation}
			\oint E_n\d{S}=\frac{q}{\varepsilon_0},
		\end{equation}
	где $\varepsilon_0$ -- величина, называемая диэлектрической проницаемостию вакуума \index{Диэлектрическая проницаемость!вакуума}. Пропорцию~(\ref{eq:prop1}) можно записать в~таком виде, поскольку $E\prop\dfrac{1}{R^2}$, $S\prop R^2$, а~тогда $ES=\c\prop q$.\par
	Рассмотрим теперь простую сферическую поверхность \index{Поверхность!сферическая} с зарядом внутри. Тогда
		$$\Phi_E=4\pi R^2 E=\frac{q}{\varepsilon_0},$$
	откуда
		$$E=\frac{1}{4\pi\varepsilon_0}\frac{q}{R^2}.$$
	Но $E=k\dfrac{q}{R^2}$, значит,
		\begin{equation}
			k=\frac{1}{4\pi\varepsilon_0}.
		\end{equation}
	Отсюда $\varepsilon_0=8,85\times 10^{-12} \dfrac{\text{Кл}^2}{\text{Н}\cdot\text{м}}$.